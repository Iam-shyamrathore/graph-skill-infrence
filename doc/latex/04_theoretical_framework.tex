\section{Theoretical Framework: Skill Inference Information Network (SIIN)}
\label{sec:theoretical_framework}

\subsection{Introduction}
\subsection{System Axioms}
The SIIN framework is built upon three core axioms:
\begin{enumerate}
    \item \textbf{Semantic Continuity}: Technical identity is a structural property of a developer's position within a Heterogeneous Information Network (HIN).
    \item \textbf{Rational Exploration}: Graph traversal is modeled as a Markov Decision Process (MDP) where rewards prioritize novelty and reasoning clarity over simple hit-rates.
    \item \textbf{Honest Uncertainty}: The system must distinguish between evidence of absence and the absence of evidence through rigorous Subjective Logic intervals.
\end{enumerate}

\subsection{The SIIN Algorithm: MCTS-SL Fusion}
The inference process operates in a three-stage pipeline:
\begin{enumerate}
    \item \textbf{HIN Construction}: A schema $T_G = \{D, P, C, F, S, I\}$ is populated using TF-IDF weighted semantic diffs and repository metadata.
    \item \textbf{DeepPath Exploration}: A path-aware MCTS agent explores the HIN using a multi-faceted reward function $R = R_{acc} + R_{eff} + R_{div}$ and Chain-of-Thought reasoning.
    \item \textbf{Subjective Logic Fusion}: Evidence paths are treated as trust-chains. Final confidence values are derived through path-discounting ($\otimes$) and Yager-consistent consensus ($\oplus$).
\end{enumerate}

\subsection{Formal Contribution}
Unlike traditional profilers that rely on flat keyword frequency, SIIN provides:
\begin{itemize}
    \item \textbf{Contextual Reasoning}: LLM-backed causal explanations for every inferred skill.
    \item \textbf{Conflict Awareness}: Identifying disagreement between repository sources via reallocated uncertainty.
    \item \textbf{Structural Validation}: Using PathSim normalization and trust propogation to ground skills in developer visibility.
\end{itemize}
